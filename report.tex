% Created 2019-09-08 Вс 23:39
% Intended LaTeX compiler: pdflatex
\documentclass[11pt]{article}
\usepackage[utf8]{inputenc}
\usepackage[T1]{fontenc}
\usepackage{graphicx}
\usepackage{grffile}
\usepackage{longtable}
\usepackage{wrapfig}
\usepackage{rotating}
\usepackage[normalem]{ulem}
\usepackage{amsmath}
\usepackage{textcomp}
\usepackage{amssymb}
\usepackage{capt-of}
\usepackage{hyperref}
\usepackage[T2A]{fontenc}
\usepackage[a4paper,left=3cm,top=2cm,right=1.5cm,bottom=2cm,marginparsep=7pt,marginparwidth=.6in]{geometry}
\usepackage{cmap}
\usepackage[russian, english]{babel}
\usepackage{xcolor}
\usepackage{listings}
\author{Крутько Никита}
\date{\today}
\title{}
\hypersetup{
 pdfauthor={Крутько Никита},
 pdftitle={},
 pdfkeywords={},
 pdfsubject={},
 pdfcreator={Emacs 26.1 (Org mode 9.1.9)}, 
 pdflang={English}}
\begin{document}

\large
\thispagestyle{empty}
\begin{center}
\textbf{Санкт-Петербургский Национальный Исследовательский}\\
\textbf{Университет Информационных Технологий, Механики и Оптики}\\
\textbf{Факультет Программной Инженерии и Компьютерной Техники}\\
\end{center}
\vspace{1em}
\begin{center}
\includegraphics[width=120px]{/home/krutna/.emacs.d/org-mode/itmo/itmo-logo.png}
\end{center}
\LARGE
\vspace{5em}
\begin{center}
\textbf{Вариант №}\\
\textbf{Лабораторная работа № 1}\\
\Large
\textbf{по дисциплине}\\
\LARGE
\textbf{\emph{'Программирование'}}\\
\end{center}
\vspace{11em}
\large
\begin{flushright}
\textbf{Выполнил:}\\
\textbf{Студент группы P3113}\\
\textbf{\emph{Крутько Никита} : 242570}\\
\textbf{Преподаватель:}\\
\textbf{\emph{Письмак Алексей Евгеньевич}}\\
\end{flushright}
\vspace{4em}
\large
\begin{center}
\textbf{Санкт-Петербург 2019 г.}
\end{center}
\pagebreak{}
\setcounter{tocdepth}{2}
\tableofcontents
\vspace{2em}
\section{Задание:}
\label{sec:org7429683}
Написать программу на языке \textbf{Java}, выполняющую соответствующие варианту действия. Программа должна соответствовать следующим требованиям:
\begin{enumerate}
\item Она должна быть упакована в исполняемый jar-архив.
\item Выражение должно вычисляться в соответствии с правилами вычисления матема\-тических выражений (должен соблюдаться порядок выполнения действий и т.д.).
\item Программа должна использовать математические функции из стандартной биб\-лиотеки Java.
\item Результат вычисления выражения должен быть выведен в стандартный поток вывода в заданном формате.
\end{enumerate}
Выполнение программы необходимо продемонстрировать на сервере \textbf{helios}.

\vspace{2em}
\section{Исходный код:}
\label{sec:orgcf525e7}
\subsection{Main.java:}
\label{sec:org6500b0c}
\small
\lstset{language=Java,label= ,caption= ,captionpos=b,numbers=none}
\begin{lstlisting}
import java.lang.Math;
import java.util.stream.IntStream;

public class Main {

    public static void main(String[] args) {    
	// Setup constants for first array
	final int firstArraySize = 18;
	final int firstArrayMaxValue = 19;
	// Setup constants for second array
	final int secondArraySize = 11;
	final float secondArrayMin = -15.0F;
	final float secondArrayMax = 9.0F;
	// Setup constants for third array
	final int firstFunctionExpectedValue = 15;
	final int[] secondFunctionExpectedArray = {
	    2, 7, 8, 10, 11,
	    12, 14, 16, 19
	};

	// Arrays initialization
	int[] first = new int[firstArraySize];
	float[] second = new float[secondArraySize];
	double[][] third = new double[firstArraySize][secondArraySize];

	// Setup values of first array
	for (int i = 0; i < firstArraySize; ++i) {
	    first[i] = firstArrayMaxValue - i;
	}
	// Setup values of second array
	for (int i = 0; i < secondArraySize; ++i) {
	    second[i] = (float)
		(Math.random() *
		 (secondArrayMax + secondArrayMin) -
		 secondArrayMin);

	}
	// Setup values of third array
	for (int i = 0; i < firstArraySize; ++i) {
	    for (int j = 0; j < secondArraySize; ++j) {
		final int value = first[i];
		final float result = second[j];
		if (value == firstFunctionExpectedValue) {

		    third[i][j] = firstFunction(result);

		} else if (IntStream.of(secondFunctionExpectedArray)
			   .anyMatch(val -> val == value)) {

		    third[i][j] = secondFunction(result);

		} else {

		    third[i][j] = thirdFunction(result);

		}
	    }
	}
	// Print third function to the screen
	// printToScreen(third, firstArraySize, secondArraySize);
	printFullToScreen(third,
			  first, firstArraySize,
			  second, secondArraySize);
    }

    static double firstFunction(final double value) {
	return Math.tan(Math.cos(powAnyValue(value,
					     (2.0/3.0) * value)));
    }

    static double secondFunction(final double value) {
	double dimension = 0.75 / (0.25 - Math.asin((value - 3.0) / 24.0));
	return powAnyValue((2.0 + powAnyValue(Math.tan(value),
					    Math.cos(value))) / 2.0,
			   dimension);
    }

    static double thirdFunction(final double value) {
	return (1.0/3.0) /
	    (powAnyValue((4.0 /
			  Math.tan((powAnyValue(value,
						(value + 1.0) / value)))),
			 3.0) - 4.0);
    }

    static double powAnyValue(double value, final double dimension) {
	boolean isNegative = value < 0;
	if (isNegative) {
	    value *= -1.0;
	}
	double result = Math.pow(value, dimension);
	if (isNegative) {
	    result *= -1.0;
	}
	return result;
    }

    private static void
	printToScreen(final double[][] arr,
		      final int columns,
		      final int lines) {
	for (int i = 0; i < columns; ++i) {
	    for (int j = 0; j < lines; ++j) {
		System.out.printf("%13.4e", arr[i][j]);
	    }
	    System.out.println();
	}
    }
    private static void
	printFullToScreen(final double[][] arr,
			  final int[] first,
			  final int lines,
			  final float[] second,
			  final int columns) {
	System.out.printf("    \u2503");
	for (int i = 0; i < columns; ++i) {
	    System.out.printf("%13.4e", second[i]);
	}
	System.	out.println();
	for (int i = 0; i < columns * 13 + 5; ++i) {
	    if (i == 4)
		System.out.printf("\u254B");
	    else
		System.out.printf("\u2501");
	}
	System.out.println();
	for (int i = 0; i < lines; ++i) {
	    System.out.printf("%3d \u2503", first[i]);
	    for (int j = 0; j < columns; ++j) {
		System.out.printf("%13.4e", arr[i][j]);
	    }
	    System.out.println();
	}
    }

}
\end{lstlisting}
\pagebreak{} \large
\section{Вывод:}
\label{sec:org86b5a10}
Я познакомился с языком \textbf{Java}, изучил некоторые библиотеки (\textbf{Math}) и методы классов и областей имён, 
\end{document}